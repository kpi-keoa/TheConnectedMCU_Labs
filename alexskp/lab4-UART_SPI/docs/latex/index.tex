lab4-\/\+U\+A\+R\+T\+\_\+\+S\+PI is a university laboratory project for P\+I\+C32 Wi\+Fire kit written in C language for studying basic configuring and usage of U\+A\+RT and S\+PI protocols.

In project was implemented controlling an angle of the servo through U\+A\+RT and displaying setted angle of the servo on O\+L\+ED S\+PI display. Using C\+OM port terminal you can turn servo to one of three defined angles (0, 90, 180) by sending \textquotesingle{}1\textquotesingle{}, \textquotesingle{}2\textquotesingle{} or \textquotesingle{}3\textquotesingle{} respectively. It uses 115200 baud rate.

Project was documented using doxygen, so if you have \char`\"{}latex\char`\"{} on your machine you can simply generate project\textquotesingle{}s reference manual in pdf by using \char`\"{}make pdf\char`\"{} command in your console.

 